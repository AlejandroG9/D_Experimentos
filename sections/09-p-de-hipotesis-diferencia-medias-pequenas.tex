%! Author = alejandrogonzalezturrubiates
%! Date = 09/09/25

\section{Pruebas de hipótesis para diferencia entre medias (muestras pequeñas)}

Cuando los tamaños de muestra son \textbf{pequeños} ($n_1, n_2 < 30$) y las varianzas poblacionales son \textbf{desconocidas pero homogéneas}, se utiliza la distribución $t$ con varianza combinada (pooled):

\[
S_p^2 = \frac{(n_1-1)S_1^2 + (n_2-1)S_2^2}{n_1+n_2-2},
\qquad
SE = S_p \sqrt{\tfrac{1}{n_1} + \tfrac{1}{n_2}}.
\]

El estadístico es
\[
t = \frac{(\bar{X}_1 - \bar{X}_2)-\Delta_0}{SE},
\qquad t \sim t_{gl}, \quad gl=n_1+n_2-2.
\]

% ============================================================
\begin{ejercicio}[Sensibilidad de gerentes con programas educativos]
Una compañía ha estado investigando dos programas educativos para mejorar la \textbf{sensibilidad de sus gerentes} hacia la necesidad de los empleados.
El programa \textbf{formal} incluye contacto en aula con psicólogos y sociólogos profesionales.
El programa \textbf{informal} consiste en sesiones de preguntas y respuestas.

El presidente desea probar, con $\alpha=0.05$, si el nuevo programa formal (más costoso) realmente mejora la sensibilidad.

\begin{center}
\begin{tabular}{@{}lccc@{}}
\toprule
 & $\bar{x}$ & $s$ & $n$ \\
\midrule
Formal & 92 & 15 & 12 \\
Informal & 84 & 19 & 15 \\
\bottomrule
\end{tabular}
\end{center}

\textbf{Actividad}
\begin{pasos}
  \item Plantee $H_0:\mu_1-\mu_2=0$ \quad vs. \quad $H_1:\mu_1-\mu_2>0$ (unilateral).
  \item Calcule la varianza combinada:
  \[
  S_p^2 = \frac{(12-1)(15^2)+(15-1)(19^2)}{12+15-2}.
  \]
  \item Calcule el error estándar:
  \[
  SE = S_p \sqrt{\tfrac{1}{12}+\tfrac{1}{15}}.
  \]
  \item Obtenga el estadístico:
  \[
  t_{\text{obs}} = \frac{(92-84)-0}{SE}.
  \]
  \item Compare con $t_{0.05,\,gl=25}$ y concluya si se rechaza $H_0$.
\end{pasos}

\begin{clave}
\noindent \textbf{Resultados de referencia:}\\
$S_p^2 \approx 289.1$, $S_p\approx 17.0$, $SE\approx 6.65$.
$t_{\text{obs}} \approx 1.20$, $gl=25$, $t_{0.05,25}\approx 1.708$.
Como $1.20 < 1.708$, \textbf{no se rechaza $H_0$}. No hay evidencia suficiente al 5\% de que el programa formal sea superior.
\end{clave}
\end{ejercicio}


% ============================================================
\begin{ejercicio}[Productividad entre dos grupos de operarios]
Una planta industrial implementa dos programas de capacitación distintos. Se desea saber si el programa A incrementa la productividad (unidades/hora) respecto al programa B.

\begin{center}
\begin{tabular}{@{}lccc@{}}
\toprule
 & $\bar{x}$ & $s$ & $n$ \\
\midrule
Programa A & 55 & 6 & 10 \\
Programa B & 52 & 5 & 12 \\
\bottomrule
\end{tabular}
\end{center}

\textbf{Actividad}
\begin{pasos}
  \item Plantee $H_0:\mu_A-\mu_B=0$ vs. $H_1:\mu_A-\mu_B>0$.
  \item Calcule $S_p^2$ y $SE$.
  \item Obtenga $t_{\text{obs}}$.
  \item Compare con $t_{0.05,\,gl=20}$.
  \item Interprete si el programa A es más efectivo.
\end{pasos}

\begin{clave}
$S_p^2\approx30.0$, $S_p\approx5.48$, $SE\approx2.33$, $t_{\text{obs}}\approx1.29$; $t_{0.05,20}=1.725$.
Conclusión: no se rechaza $H_0$.
\end{clave}
\end{ejercicio}

% ============================================================
\begin{ejercicio}[Resistencia de materiales de dos proveedores]
Se comparan las resistencias (MPa) de acero de dos proveedores.

\begin{center}
\begin{tabular}{@{}lccc@{}}
\toprule
 & $\bar{x}$ & $s$ & $n$ \\
\midrule
Proveedor A & 410 & 25 & 14 \\
Proveedor B & 395 & 22 & 12 \\
\bottomrule
\end{tabular}
\end{center}

\textbf{Actividad}
\begin{pasos}
  \item Hipótesis: $H_0:\mu_A-\mu_B=0$ vs. $H_1:\mu_A-\mu_B\neq0$.
  \item Calcule $S_p^2$ y $SE$.
  \item Calcule $t_{\text{obs}}$ y compárelo con $t_{0.025,24}$.
\end{pasos}

\begin{clave}
$S_p\approx23.7$, $SE\approx9.0$, $t_{\text{obs}}\approx1.67$, $t_{0.025,24}=2.064$.
Conclusión: no se rechaza $H_0$.
\end{clave}
\end{ejercicio}

% ============================================================
\begin{ejercicio}[Tiempo de ensamblaje en dos turnos]
Se quiere comprobar si el turno de la mañana tarda menos que el turno de la tarde en un proceso de ensamblaje.

\begin{center}
\begin{tabular}{@{}lccc@{}}
\toprule
 & $\bar{x}$ & $s$ & $n$ \\
\midrule
Mañana & 48.5 & 4.2 & 11 \\
Tarde & 51.0 & 5.0 & 10 \\
\bottomrule
\end{tabular}
\end{center}

\textbf{Actividad}
\begin{pasos}
  \item Hipótesis: $H_0:\mu_M-\mu_T=0$ vs. $H_1:\mu_M-\mu_T<0$.
  \item Calcule $S_p^2$, $SE$, y $t_{\text{obs}}$.
  \item Compare con $t_{0.05,19}$.
\end{pasos}

\begin{clave}
$S_p\approx4.58$, $SE\approx2.0$, $t_{\text{obs}}\approx-1.25$, $t_{0.05,19}=-1.729$.
Conclusión: no se rechaza $H_0$.
\end{clave}
\end{ejercicio}

% ============================================================
\begin{ejercicio}[Satisfacción de clientes en dos sucursales]
Se comparan las calificaciones promedio (escala 0--100) de dos sucursales para evaluar si existen diferencias significativas.

\begin{center}
\begin{tabular}{@{}lccc@{}}
\toprule
 & $\bar{x}$ & $s$ & $n$ \\
\midrule
Sucursal A & 82 & 6 & 9 \\
Sucursal B & 78 & 5 & 10 \\
\bottomrule
\end{tabular}
\end{center}

\textbf{Actividad}
\begin{pasos}
  \item Hipótesis: $H_0:\mu_A-\mu_B=0$ vs. $H_1:\mu_A-\mu_B\neq0$.
  \item Calcule $S_p^2$, $SE$, y $t_{\text{obs}}$.
  \item Compare con $t_{0.025,17}$.
\end{pasos}

\begin{clave}
$S_p\approx5.48$, $SE\approx2.44$, $t_{\text{obs}}\approx1.64$, $t_{0.025,17}=2.110$.
Conclusión: no se rechaza $H_0$.
\end{clave}
\end{ejercicio}

% ============================================================
\begin{ejercicio}[Consumo de combustible de dos vehículos]
Un investigador quiere comparar el rendimiento (km/L) de dos modelos de automóvil bajo condiciones similares.

\begin{center}
\begin{tabular}{@{}lccc@{}}
\toprule
 & $\bar{x}$ & $s$ & $n$ \\
\midrule
Modelo X & 15.2 & 1.1 & 8 \\
Modelo Y & 14.0 & 1.3 & 9 \\
\bottomrule
\end{tabular}
\end{center}

\textbf{Actividad}
\begin{pasos}
  \item Hipótesis: $H_0:\mu_X-\mu_Y=0$ vs. $H_1:\mu_X-\mu_Y>0$.
  \item Calcule $S_p^2$, $SE$, y $t_{\text{obs}}$.
  \item Compare con $t_{0.05,15}$.
\end{pasos}

\begin{clave}
$S_p\approx1.21$, $SE\approx0.57$, $t_{\text{obs}}\approx2.11$, $t_{0.05,15}=1.753$.
Conclusión: se rechaza $H_0$. Evidencia de que el Modelo X rinde más.
\end{clave}
\end{ejercicio}