%! Author = alejandrogonzalezturrubiates
%! Date = 09/09/25

% ============================================================
% ============================================================
\begin{table}[H]
\centering
\renewcommand{\arraystretch}{1.4}
\begin{tabular}{p{0.30\textwidth} p{0.50\textwidth} c}
\toprule
\textbf{Criterio} & \textbf{Descripción} & \textbf{Puntos} \\
\midrule
Planteamiento de hipótesis & Define correctamente $H_0$ y $H_1$ según el contexto (una o dos colas). & 15 \\
Cálculo de proporciones muestrales & Obtiene correctamente $\hat{p}_1$, $\hat{p}_2$ y la proporción combinada $\hat{p}$. & 15 \\
Determinación del error estándar & Calcula adecuadamente $SE=\sqrt{\hat{p}(1-\hat{p})(\tfrac{1}{n_1}+\tfrac{1}{n_2})}$. & 15 \\
Cálculo del estadístico $z$ & Aplica la fórmula $z=\tfrac{(\hat{p}_1-\hat{p}_2)-\Delta_0}{SE}$ con precisión. & 15 \\
Regla de decisión & Compara $z_{\text{obs}}$ con el valor crítico (según $\alpha$ y tipo de prueba). & 15 \\
Interpretación de resultados & Conclusión clara y contextualizada (ej. diferencias en productos, servicios, etc.). & 15 \\
Presentación y orden & Claridad en procedimientos, notación, redacción y justificación. & 10 \\
\bottomrule
\end{tabular}
\caption{Rúbrica de evaluación para pruebas de hipótesis con proporciones (100 puntos).}
\end{table}