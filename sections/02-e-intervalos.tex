\section{Ejercicios de estimación por intervalos}


\begin{ejercicio}[Tiempo de \emph{setup} en prensa (n=40)]
En una celda de manufactura se desea estimar el \textbf{tiempo promedio} del proceso: \emph{Tiempo de \emph{setup} en prensa}.
Se toma una muestra de tamaño $n=40$ y se solicita construir un \textbf{IC al 95\%} para la media.

\datos{%
12.7, 11.9, 13.4, 12.1, 12.8, 11.6, 14.0, 12.5, 13.1, 12.0,
11.8, 12.9, 13.3, 12.2, 11.7, 12.6, 13.0, 12.4, 11.5, 12.3,
13.2, 12.7, 11.9, 12.8, 13.5, 12.6, 12.0, 11.8, 12.1, 13.1,
12.5, 11.6, 12.9, 13.0, 12.4, 11.7, 12.3, 13.4, 12.2, 11.9
}

\textbf{Actividad}
\begin{pasos}
  \item Calcule $\bar{x}$ y $s$.
  \item Calcule $SE = s/\sqrt{n}$.
  \item Seleccione el valor crítico (use $t$ con $gl=n-1$ y compare con $Z=1.96$).
  \item Obtenga $ME=\text{crítico}\times SE$ e informe el IC $\bar{x}\pm ME$.
  \item Interprete los resultados para la toma de decisiones (p.ej., SMED, balanceo, estandarización).
\end{pasos}

\begin{clave}
\noindent Valores de referencia (usando $Z=1.96$):\\
$\bar{x} \approx 12.485$, \quad $s \approx 0.618$, \quad $SE \approx 0.098$,\\
$ME \approx 0.191$, \quad IC95\% $\approx (12.294,\, 12.676)$.
\end{clave}
\end{ejercicio}


\begin{ejercicio}[Tiempo de inspección final por lote (n=50)]
En una celda de manufactura se desea estimar el \textbf{tiempo promedio} del proceso: \emph{Tiempo de inspección final por lote}.
Se toma una muestra de tamaño $n=50$ y se solicita construir un \textbf{IC al 95\%} para la media.

\datos{%
12.2, 12.6, 11.8, 12.7, 12.9, 12.4, 11.9, 12.2, 12.8, 12.3,
12.3, 11.6, 11.9, 13.7, 12.3, 12.3, 12.9, 12.0, 13.2, 12.5,
12.4, 12.1, 12.4, 13.5, 12.8, 12.5, 12.7, 12.7, 12.3, 11.8,
12.1, 13.0, 12.2, 12.5, 13.5, 12.7, 12.9, 12.2, 13.0, 11.4,
13.2, 12.0, 12.7, 11.9, 12.1, 12.6, 13.8, 11.4, 12.6, 13.3
}

\textbf{Actividad}
\begin{pasos}
  \item Calcule $\bar{x}$ y $s$.
  \item Calcule $SE = s/\sqrt{n}$.
  \item Seleccione el valor crítico (use $t$ con $gl=n-1$ y compare con $Z=1.96$).
  \item Obtenga $ME=\text{crítico}\times SE$ e informe el IC $\bar{x}\pm ME$.
  \item Interprete los resultados para la toma de decisiones (p.ej., SMED, balanceo, estandarización).
\end{pasos}

\begin{clave}
\noindent Valores de referencia (usando $Z=1.96$):\\
$\bar{x} \approx 12.496$, \quad $s \approx 0.553$, \quad $SE \approx 0.078$,\\
$ME \approx 0.153$, \quad IC95\% $\approx (12.343,\, 12.649)$.
\end{clave}
\end{ejercicio}


\begin{ejercicio}[Tiempo de cambio de herramienta CNC (n=39)]
En una celda de manufactura se desea estimar el \textbf{tiempo promedio} del proceso: \emph{Tiempo de cambio de herramienta CNC}.
Se toma una muestra de tamaño $n=39$ y se solicita construir un \textbf{IC al 95\%} para la media.

\datos{%
12.2, 11.2, 12.4, 12.6, 11.8, 12.5, 12.5, 12.6, 12.5, 11.3,
12.2, 12.9, 12.8, 11.8, 12.0, 12.2, 12.8, 12.8, 12.3, 11.0,
12.9, 13.3, 13.4, 12.5, 12.8, 12.3, 12.2, 12.5, 12.6, 13.1,
13.6, 12.6, 12.6, 11.8, 12.2, 11.4, 11.9, 12.8, 11.6
}

\textbf{Actividad}
\begin{pasos}
  \item Calcule $\bar{x}$ y $s$.
  \item Calcule $SE = s/\sqrt{n}$.
  \item Seleccione el valor crítico (use $t$ con $gl=n-1$ y compare con $Z=1.96$).
  \item Obtenga $ME=\text{crítico}\times SE$ e informe el IC $\bar{x}\pm ME$.
  \item Interprete los resultados para la toma de decisiones (p.ej., SMED, balanceo, estandarización).
\end{pasos}

\begin{clave}
\noindent Valores de referencia (usando $Z=1.96$):\\
$\bar{x} \approx 12.372$, \quad $s \approx 0.590$, \quad $SE \approx 0.094$,\\
$ME \approx 0.185$, \quad IC95\% $\approx (12.187,\, 12.557)$.
\end{clave}
\end{ejercicio}


\begin{ejercicio}[Tiempo de transporte interno por pallet (n=46)]
En una celda de manufactura se desea estimar el \textbf{tiempo promedio} del proceso: \emph{Tiempo de transporte interno por pallet}.
Se toma una muestra de tamaño $n=46$ y se solicita construir un \textbf{IC al 95\%} para la media.

\datos{%
12.3, 13.7, 13.0, 11.6, 11.9, 12.8, 12.7, 12.4, 12.4, 12.7,
13.1, 12.4, 13.4, 13.3, 12.1, 13.0, 12.2, 12.2, 12.9, 12.8,
10.9, 12.4, 12.9, 11.2, 13.2, 12.7, 13.1, 11.8, 12.8, 12.7,
12.2, 11.6, 12.6, 13.3, 12.3, 12.8, 11.2, 12.8, 12.2, 12.4,
11.8, 13.6, 12.0, 12.3, 12.8, 12.4
}

\textbf{Actividad}
\begin{pasos}
  \item Calcule $\bar{x}$ y $s$.
  \item Calcule $SE = s/\sqrt{n}$.
  \item Seleccione el valor crítico (use $t$ con $gl=n-1$ y compare con $Z=1.96$).
  \item Obtenga $ME=\text{crítico}\times SE$ e informe el IC $\bar{x}\pm ME$.
  \item Interprete los resultados para la toma de decisiones (p.ej., SMED, balanceo, estandarización).
\end{pasos}

\begin{clave}
\noindent Valores de referencia (usando $Z=1.96$):\\
$\bar{x} \approx 12.498$, \quad $s \approx 0.621$, \quad $SE \approx 0.091$,\\
$ME \approx 0.179$, \quad IC95\% $\approx (12.318,\, 12.677)$.
\end{clave}
\end{ejercicio}


\begin{ejercicio}[Tiempo de verificación dimensional (n=45)]
En una celda de manufactura se desea estimar el \textbf{tiempo promedio} del proceso: \emph{Tiempo de verificación dimensional}.
Se toma una muestra de tamaño $n=45$ y se solicita construir un \textbf{IC al 95\%} para la media.

\datos{%
13.2, 11.8, 12.6, 13.1, 11.5, 13.0, 13.4, 12.9, 12.4, 12.9,
13.5, 12.4, 11.8, 12.6, 11.7, 12.2, 13.2, 13.3, 12.3, 13.0,
13.1, 12.0, 12.8, 10.9, 12.4, 12.0, 12.2, 12.0, 13.4, 12.1,
12.2, 12.6, 12.5, 12.5, 13.4, 13.9, 12.9, 12.8, 11.8, 13.0,
12.1, 12.4, 13.0, 13.3, 13.2
}

\textbf{Actividad}
\begin{pasos}
  \item Calcule $\bar{x}$ y $s$.
  \item Calcule $SE = s/\sqrt{n}$.
  \item Seleccione el valor crítico (use $t$ con $gl=n-1$ y compare con $Z=1.96$).
  \item Obtenga $ME=\text{crítico}\times SE$ e informe el IC $\bar{x}\pm ME$.
  \item Interprete los resultados para la toma de decisiones (p.ej., SMED, balanceo, estandarización).
\end{pasos}

\begin{clave}
\noindent Valores de referencia (usando $Z=1.96$):\\
$\bar{x} \approx 12.607$, \quad $s \approx 0.624$, \quad $SE \approx 0.093$,\\
$ME \approx 0.182$, \quad IC95\% $\approx (12.424,\, 12.789)$.
\end{clave}
\end{ejercicio}


\begin{ejercicio}[Tiempo de soldadura por punto (n=47)]
En una celda de manufactura se desea estimar el \textbf{tiempo promedio} del proceso: \emph{Tiempo de soldadura por punto}.
Se toma una muestra de tamaño $n=47$ y se solicita construir un \textbf{IC al 95\%} para la media.

\datos{%
12.2, 13.4, 12.0, 12.5, 12.2, 12.8, 12.0, 13.5, 13.2, 13.1,
11.6, 11.9, 12.6, 12.6, 11.6, 12.5, 11.2, 12.5, 11.8, 12.9,
12.3, 12.6, 12.1, 13.1, 12.8, 12.5, 12.5, 12.2, 13.3, 12.6,
12.1, 12.3, 12.0, 12.9, 11.8, 12.4, 12.8, 12.6, 12.2, 12.2,
12.8, 12.7, 12.8, 12.9, 13.0, 12.3, 12.7
}

\textbf{Actividad}
\begin{pasos}
  \item Calcule $\bar{x}$ y $s$.
  \item Calcule $SE = s/\sqrt{n}$.
  \item Seleccione el valor crítico (use $t$ con $gl=n-1$ y compare con $Z=1.96$).
  \item Obtenga $ME=\text{crítico}\times SE$ e informe el IC $\bar{x}\pm ME$.
  \item Interprete los resultados para la toma de decisiones (p.ej., SMED, balanceo, estandarización).
\end{pasos}

\begin{clave}
\noindent Valores de referencia (usando $Z=1.96$):\\
$\bar{x} \approx 12.481$, \quad $s \approx 0.495$, \quad $SE \approx 0.072$,\\
$ME \approx 0.141$, \quad IC95\% $\approx (12.339,\, 12.622)$.
\end{clave}
\end{ejercicio}
