%! Author = alejandrogonzalezturrubiates
%! Date = 25/08/25

\section{Ejercicios de estimación de proporciones}

\begin{ejercicio}[Defectuosos en producción (n=200)]
En una muestra de $n=200$ tornillos producidos, se detectaron $x=18$ defectuosos.
Se desea estimar la \textbf{proporción de defectuosos} con un IC al 95\%.

\datos{%
$n=200$, \quad $x=18$
}

\textbf{Actividad}
\begin{pasos}
  \item Calcule $\hat{p} = x/n$.
  \item Calcule $SE = \sqrt{\hat{p}(1-\hat{p})/n}$.
  \item Seleccione el valor crítico ($Z=1.96$).
  \item Obtenga $ME=Z \times SE$ e informe el IC $\hat{p}\pm ME$.
  \item Interprete el intervalo en el contexto de la calidad de producción.
\end{pasos}

\begin{clave}
$\hat{p}=0.090$, \quad $SE\approx0.020$, \quad $ME\approx0.039$,\\
IC95\% $\approx (0.051,\,0.129)$.
\end{clave}
\end{ejercicio}


\begin{ejercicio}[Envases con fuga (n=150)]
En una muestra de $n=150$ envases, $x=24$ presentaron fugas.
Se desea estimar la \textbf{proporción de envases defectuosos} con un IC al 95\%.

\datos{%
$n=150$, \quad $x=24$
}

\textbf{Actividad}
\begin{pasos}
  \item Calcule $\hat{p} = x/n$.
  \item Calcule $SE = \sqrt{\hat{p}(1-\hat{p})/n}$.
  \item Seleccione el valor crítico ($Z=1.96$).
  \item Obtenga $ME=Z \times SE$ e informe el IC $\hat{p}\pm ME$.
  \item Interprete el intervalo en el contexto de control de envases.
\end{pasos}

\begin{clave}
$\hat{p}=0.160$, \quad $SE\approx0.030$, \quad $ME\approx0.059$,\\
IC95\% $\approx (0.101,\,0.219)$.
\end{clave}
\end{ejercicio}


\begin{ejercicio}[Uso de EPP (n=80)]
Se observa el uso de equipo de protección personal en $n=80$ empleados y $x=68$ lo portaban correctamente.
Se desea estimar la \textbf{proporción de cumplimiento} con un IC al 95\%.

\datos{%
$n=80$, \quad $x=68$
}

\textbf{Actividad}
\begin{pasos}
  \item Calcule $\hat{p} = x/n$.
  \item Calcule $SE = \sqrt{\hat{p}(1-\hat{p})/n}$.
  \item Seleccione el valor crítico ($Z=1.96$).
  \item Obtenga $ME=Z \times SE$ e informe el IC $\hat{p}\pm ME$.
  \item Interprete el intervalo en el contexto de seguridad laboral.
\end{pasos}

\begin{clave}
$\hat{p}=0.850$, \quad $SE\approx0.040$, \quad $ME\approx0.078$,\\
IC95\% $\approx (0.772,\,0.928)$.
\end{clave}
\end{ejercicio}


\begin{ejercicio}[Rechazo de lotes (n=120)]
En una auditoría de materia prima con $n=120$ lotes, se rechazaron $x=15$.
Se desea estimar la \textbf{proporción de rechazo} con un IC al 95\%.

\datos{%
$n=120$, \quad $x=15$
}

\textbf{Actividad}
\begin{pasos}
  \item Calcule $\hat{p} = x/n$.
  \item Calcule $SE = \sqrt{\hat{p}(1-\hat{p})/n}$.
  \item Seleccione el valor crítico ($Z=1.96$).
  \item Obtenga $ME=Z \times SE$ e informe el IC $\hat{p}\pm ME$.
  \item Interprete el intervalo en el contexto de control de calidad.
\end{pasos}

\begin{clave}
$\hat{p}=0.125$, \quad $SE\approx0.030$, \quad $ME\approx0.059$,\\
IC95\% $\approx (0.066,\,0.184)$.
\end{clave}
\end{ejercicio}


\begin{ejercicio}[Entregas a tiempo (n=250)]
De $n=250$ pedidos, $x=223$ llegaron a tiempo.
Se desea estimar la \textbf{proporción de entregas puntuales} con un IC al 95\%.

\datos{%
$n=250$, \quad $x=223$
}

\textbf{Actividad}
\begin{pasos}
  \item Calcule $\hat{p} = x/n$.
  \item Calcule $SE = \sqrt{\hat{p}(1-\hat{p})/n}$.
  \item Seleccione el valor crítico ($Z=1.96$).
  \item Obtenga $ME=Z \times SE$ e informe el IC $\hat{p}\pm ME$.
  \item Interprete el intervalo en el contexto de logística y entregas.
\end{pasos}

\begin{clave}
$\hat{p}=0.892$, \quad $SE\approx0.020$, \quad $ME\approx0.039$,\\
IC95\% $\approx (0.853,\,0.931)$.
\end{clave}
\end{ejercicio}