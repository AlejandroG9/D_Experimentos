%! Author = alejandrogonzalezturrubiates
%! Date = 09/09/25

% ============================================================
\section{Pruebas de hipótesis para proporciones (muestras grandes)}

Sea $\hat{p}_1$ y $\hat{p}_2$ la proporción de éxito en dos muestras independientes, de tamaños $n_1$ y $n_2$.
Se desea contrastar:
\[
H_0: p_1-p_2=0
\qquad\text{vs}\qquad
H_1: p_1-p_2 \ \text{(}\neq,>,<\text{)}\ 0.
\]

El estadístico es
\[
Z \;=\; \frac{(\hat{p}_1-\hat{p}_2)-0}{\sqrt{\hat{p}(1-\hat{p})\left(\tfrac{1}{n_1}+\tfrac{1}{n_2}\right)}},
\]
donde $\hat{p}$ es la proporción combinada:
\[
\hat{p} = \frac{n_1\hat{p}_1+n_2\hat{p}_2}{n_1+n_2}.
\]

% ============================================================
\begin{ejercicio}[Comparación de eficacia de dos fármacos]
Una compañía farmacéutica prueba dos nuevas sustancias para reducir la presión sanguínea.
En el grupo 1, 71 de 100 pacientes responden al fármaco;
en el grupo 2, 58 de 90 responden.
Con $\alpha=0.05$, ¿hay diferencia significativa?

\datos{%
\begin{center}
\begin{tabular}{@{}lccc@{}}
\toprule
 & Éxitos & $n$ & $\hat{p}$ \\
\midrule
Grupo 1 & 71 & 100 & 0.71 \\
Grupo 2 & 58 & 90 & 0.644 \\
\bottomrule
\end{tabular}
\end{center}
\medskip
Nivel de significancia: $\alpha=0.05$ (bilateral).
}

\textbf{Actividad}
\begin{pasos}
  \item Plantee $H_0: p_1-p_2=0$ \;vs\; $H_1: p_1-p_2\neq 0$.
  \item Calcule la proporción combinada:
  \[
  \hat{p}=\frac{71+58}{100+90}\approx 0.6789.
  \]
  \item Calcule $SE=\sqrt{\hat{p}(1-\hat{p})(1/n_1+1/n_2)}\approx 0.0678$.
  \item Obtenga $z_{\text{obs}}=(0.71-0.644)/0.0678\approx 0.972$.
  \item Compare con $z_{0.025}=1.96$ (bilateral).
\end{pasos}

\begin{clave}
$SE\approx 0.0678$;\quad $z_{\text{obs}}\approx 0.972$;\quad $|z|<1.96$.
\textbf{No se rechaza $H_0$}. No hay evidencia de diferencia significativa entre los fármacos al 5\%.
(IC95\% opcional: $(-0.048,\,0.178)$).
\end{clave}
\end{ejercicio}



% ============================================================
\begin{ejercicio}[Defectos en producción de dos líneas]
Una planta quiere comparar la proporción de piezas defectuosas entre dos líneas de producción.
En la línea 1, 45 de 300 piezas fueron defectuosas;
en la línea 2, 70 de 400 piezas fueron defectuosas.
Con $\alpha=0.05$, ¿hay diferencia en las tasas de defectos?

\datos{%
\begin{center}
\begin{tabular}{@{}lccc@{}}
\toprule
 & Defectuosas & $n$ & $\hat{p}$ \\
\midrule
Línea 1 & 45 & 300 & 0.150 \\
Línea 2 & 70 & 400 & 0.175 \\
\bottomrule
\end{tabular}
\end{center}
Nivel de significancia: $\alpha=0.05$ (bilateral).
}

\textbf{Actividad}
\begin{pasos}
  \item Plantee $H_0:p_1-p_2=0$ vs. $H_1:p_1-p_2\neq 0$.
  \item Calcule $\hat{p}=\tfrac{45+70}{700}=0.164$.
  \item Obtenga $SE$ y $z_{\text{obs}}$.
  \item Compare con $z_{0.025}=1.96$.
\end{pasos}

\begin{clave}
$SE\approx 0.026$;\quad $z_{\text{obs}}\approx -0.96$;\quad $|z|<1.96$.
No hay diferencia significativa en defectos.
\end{clave}
\end{ejercicio}

% ============================================================
\begin{ejercicio}[Preferencia de clientes entre dos productos]
Un estudio de mercado pregunta a clientes su preferencia entre dos bebidas.
En una muestra de 250 personas en ciudad A, 180 prefieren la bebida X;
en ciudad B, 160 de 220 la prefieren.
¿Existe diferencia significativa al 1%?

\datos{%
\begin{center}
\begin{tabular}{@{}lccc@{}}
\toprule
 & Éxitos & $n$ & $\hat{p}$ \\
\midrule
Ciudad A & 180 & 250 & 0.720 \\
Ciudad B & 160 & 220 & 0.727 \\
\bottomrule
\end{tabular}
\end{center}
Nivel de significancia: $\alpha=0.01$.
}

\textbf{Actividad}
\begin{pasos}
  \item $H_0:p_1-p_2=0$ vs. $H_1:p_1-p_2\neq 0$.
  \item Calcule $\hat{p}=\tfrac{180+160}{470}\approx 0.723$.
  \item Obtenga $SE$ y $z_{\text{obs}}$.
  \item Compare con $z_{0.005}=2.576$.
\end{pasos}

\begin{clave}
$SE\approx 0.042$;\quad $z_{\text{obs}}\approx -0.17$;\quad
$|z|<2.576$, no se rechaza $H_0$.
Las preferencias son estadísticamente iguales.
\end{clave}
\end{ejercicio}

% ============================================================
\begin{ejercicio}[Cumplimiento de estándares de seguridad]
Se inspeccionan dos fábricas para ver si cumplen con estándares de seguridad.
En la fábrica 1, 90 de 120 trabajadores cumplen;
en la fábrica 2, 130 de 200 cumplen.
Con $\alpha=0.05$, ¿la proporción de cumplimiento es mayor en la fábrica 1?

\datos{%
\begin{center}
\begin{tabular}{@{}lccc@{}}
\toprule
 & Cumplen & $n$ & $\hat{p}$ \\
\midrule
Fábrica 1 & 90 & 120 & 0.750 \\
Fábrica 2 & 130 & 200 & 0.650 \\
\bottomrule
\end{tabular}
\end{center}
Nivel de significancia: $\alpha=0.05$ (una cola).
}

\textbf{Actividad}
\begin{pasos}
  \item $H_0:p_1-p_2=0$ vs. $H_1:p_1-p_2>0$.
  \item Calcule $\hat{p}=\tfrac{220}{320}=0.688$.
  \item Obtenga $SE$ y $z_{\text{obs}}$.
  \item Compare con $z_{0.05}=1.645$.
\end{pasos}

\begin{clave}
$SE\approx 0.049$;\quad $z_{\text{obs}}\approx 2.04$;\quad
$z>1.645\ \Rightarrow$ se rechaza $H_0$.
La fábrica 1 tiene mayor proporción de cumplimiento.
\end{clave}
\end{ejercicio}

% ============================================================
\begin{ejercicio}[Encuesta sobre uso de transporte público]
Se encuestan ciudadanos en dos ciudades sobre el uso de transporte público.
En ciudad A, 210 de 350 lo usan;
en ciudad B, 150 de 280 lo usan.
¿Hay diferencia al 10%?

\datos{%
\begin{center}
\begin{tabular}{@{}lccc@{}}
\toprule
 & Usan transporte & $n$ & $\hat{p}$ \\
\midrule
Ciudad A & 210 & 350 & 0.600 \\
Ciudad B & 150 & 280 & 0.536 \\
\bottomrule
\end{tabular}
\end{center}
Nivel de significancia: $\alpha=0.10$ (bilateral).
}

\textbf{Actividad}
\begin{pasos}
  \item $H_0:p_1-p_2=0$ vs. $H_1:p_1-p_2\neq 0$.
  \item Calcule $\hat{p}=\tfrac{360}{630}\approx 0.571$.
  \item Obtenga $SE$ y $z_{\text{obs}}$.
  \item Compare con $z_{0.05}=1.645$.
\end{pasos}

\begin{clave}
$SE\approx 0.040$;\quad $z_{\text{obs}}\approx 1.60$;\quad
$|z|<1.645$, no se rechaza $H_0$.
No hay diferencia significativa.
\end{clave}
\end{ejercicio}

% ============================================================
\begin{ejercicio}[Satisfacción de clientes entre dos servicios]
Una empresa mide la satisfacción de clientes (satisfecho/insatisfecho).
En servicio A, 300 de 400 están satisfechos;
en servicio B, 250 de 370 lo están.
¿Existe diferencia significativa al 5%?

\datos{%
\begin{center}
\begin{tabular}{@{}lccc@{}}
\toprule
 & Satisfechos & $n$ & $\hat{p}$ \\
\midrule
Servicio A & 300 & 400 & 0.750 \\
Servicio B & 250 & 370 & 0.676 \\
\bottomrule
\end{tabular}
\end{center}
Nivel de significancia: $\alpha=0.05$ (bilateral).
}

\textbf{Actividad}
\begin{pasos}
  \item $H_0:p_1-p_2=0$ vs. $H_1:p_1-p_2\neq 0$.
  \item Calcule $\hat{p}=\tfrac{550}{770}\approx 0.714$.
  \item Obtenga $SE$ y $z_{\text{obs}}$.
  \item Compare con $z_{0.025}=1.96$.
\end{pasos}

\begin{clave}
$SE\approx 0.031$;\quad $z_{\text{obs}}\approx 2.39$;\quad
$|z|>1.96$, se rechaza $H_0$.
Hay diferencia significativa en satisfacción.
\end{clave}
\end{ejercicio}