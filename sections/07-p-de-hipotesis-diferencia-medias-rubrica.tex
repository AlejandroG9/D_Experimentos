%! Author = alejandrogonzalezturrubiates
%! Date = 09/09/25

% ============================================================
% Rúbrica: Pruebas de hipótesis para diferencia de medias (muestras grandes)
% Archivo: sections/04-p-de-hipotesis-medias-rubrica.tex
% ============================================================

\section*{Rúbrica de evaluación: Pruebas de hipótesis para diferencia de medias}

\begin{table}[H]
\centering
\begin{tabular}{p{0.28\textwidth} p{0.52\textwidth} c}
\toprule
\textbf{Criterio} & \textbf{Descripción} & \textbf{Puntos}\\
\midrule
Planteamiento de hipótesis & Correcta formulación de $H_0$ y $H_1$ (bilateral o unilateral). & 10\\
Cálculo del error estándar (SE) & Aplicación correcta de la fórmula para $SE=\sqrt{\tfrac{\sigma_1^2}{n_1}+\tfrac{\sigma_2^2}{n_2}}$. & 15\\
Cálculo del estadístico $Z$ & Sustitución y obtención correcta de $z_{\text{obs}}$. & 15\\
Selección del valor crítico & Identificación adecuada del valor de $Z$ según $\alpha$ y tipo de prueba. & 15\\
Decisión estadística & Conclusión correcta respecto al rechazo/no rechazo de $H_0$. & 15\\
Intervalo de confianza (opcional) & Construcción del IC para $\mu_1-\mu_2$ cuando se solicite. & 10\\
Interpretación operativa & Explicación en contexto de ingeniería industrial (salarios, tiempos de ciclo, eficiencia, etc.). & 15\\
Presentación y orden & Claridad, limpieza y buena organización de resultados. & 5\\
\bottomrule
\end{tabular}
\caption{Rúbrica de evaluación: Pruebas de hipótesis para diferencia de medias (100 puntos).}
\end{table}