%! Author = alejandrogonzalezturrubiates
%! Date = 17/09/25

\documentclass[11pt,a4paper]{exam}
\usepackage[spanish]{babel}
\usepackage[utf8]{inputenc}
\usepackage{amsmath, amssymb}
\usepackage{geometry}
\geometry{margin=2cm}

\begin{document}

\begin{center}
    \Large \textbf{Examen Parcial}\\
    \large \textbf{Estimación e Inferencia Estadística}\\[0.2cm]
    \textbf{Duración: 1 hora}
\end{center}
\hrulefill

\vspace{0.5cm}
\noindent \textbf{Alumno:} \underline{\hspace{8cm}} \hfill \textbf{Grupo:} \underline{\hspace{3cm}}

\noindent \textbf{Profesor:} Dr. Alejandro González Turrubiates

\vspace{0.5cm}

% =====================
% INSTRUCCIONES
% =====================
\textbf{Instrucciones:}
Responde de forma clara y justificada cada uno de los apartados. Muéstrese el procedimiento completo: planteamiento de hipótesis, fórmulas, cálculos e interpretación.

% =====================
% SECCIÓN A: INTERVALOS DE CONFIANZA
% =====================
\begin{questions}
\section*{Sección A: Intervalos de confianza (15 pts)}

\question Se toma una muestra de $n=64$ lámparas con una media de duración $\bar{x}=730$ horas y desviación estándar conocida $\sigma=80$ horas.
\begin{parts}
    \part Construye un intervalo de confianza del 95\% para la media poblacional.
    \part Interpreta el resultado obtenido en contexto.
\end{parts}

% =====================
% SECCIÓN B: PRUEBAS DE HIPÓTESIS (MEDIA)
% =====================
\section*{Sección B: Prueba de hipótesis para la media (25 pts)}


\question Una fábrica asegura que la media de vida de sus baterías es de $1000$ horas. Un investigador toma una muestra de $n=36$, obteniendo $\bar{x}=980$ horas y $\sigma=60$ horas.
\begin{parts}
    \part Plantea las hipótesis nula y alternativa.
    \part Calcula el estadístico de prueba $z$.
    \part Con un nivel de significancia $\alpha=0.05$, determina si se rechaza $H_0$.
    \part Interpreta la decisión en contexto.
\end{parts}

% =====================
% SECCIÓN C: PROPORCIONES
% =====================
\section*{Sección C: Prueba de hipótesis para proporciones (20 pts)}

\question Se desea probar si la proporción de defectuosos en una línea de producción es diferente de 0.05. En una muestra de $n=150$ piezas, se encuentran $12$ defectuosas.
\begin{parts}
    \part Plantea $H_0$ y $H_1$.
    \part Calcula el estadístico de prueba.
    \part Concluye al 5\% de significancia.
\end{parts}

% =====================
% SECCIÓN D: MUESTRAS PEQUEÑAS
% =====================
\section*{Sección D: Prueba con muestras pequeñas (40 pts)}

\question Un estudio de $n=10$ estudiantes reporta un promedio de calificación $\bar{x}=78$ con desviación estándar $s=6$. Se quiere verificar si la media poblacional es diferente de 75 con $\alpha=0.05$.
\begin{parts}
    \part Establece las hipótesis.
    \part Calcula el estadístico $t$.
    \part Determina el valor crítico y la conclusión.
\end{parts}

\end{questions}

\vspace{1cm}
\noindent\textbf{Puntaje total: 100 puntos} \hfill \textbf{Tiempo: 1 hora}

\end{document}